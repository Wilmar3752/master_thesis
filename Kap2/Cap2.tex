\chapter{Marco teórico}
Para el tratamiento de procesos estocásticos y series de tiempo, ha sido mucha la teoría estadística implementada y se ha convertido
en una herramienta importante para tratar observaciones dependientes y entender la naturaleza de dicha dependencia,
la mayoría de estos métodos, asumen que las series de timpo son regulamente espaciadas, sin embargo este escenario
no siempre se da y ahi es donde cobran importancia los procesos estocásticos irregularmente espaciados.

Este capitulo tiene la siguiente estructura, en la seccción 2.1 presentamos la definición matematica de un proceso
estocastico irregularmente espaciado. En la sección 2.2 se presenta un proceso estocástico autorregresivo irregulamente
espaciado. En la sección 2.3 se presenta un proceso estocástico irregularmente espaciado de medias moviles y en la sección
2.3 se presenta un proceso estocástico atorregresivo de medias moviles irregulamente espaciado.


\section{Procesos estocásticos irregularmente espaciados}
Sea $(\Omega,\mathscr{B},P)$ un espacio de probabilidad. Si se define un proceso estocástico como una medida 
que mapea $x: \Omega \mapsto \mathbb{R}^{\mathbb{T}}$, donde 

$$x(\omega)= \lbrace X_{\tau}(\omega), \tau \in \mathbb{T}\rbrace$$

$\mathbb{T}$ es llamado conjunto de índices y la variable aleatoria $X_{\tau}(\omega)$ es llamada coordenada 
del proceso o trayectoria. Consideremos ahora $\mathbb{T}'= \lbrace t_1,t_2, t_3,... \rbrace$, 
con $\Delta_{n+1}=t_{n+1}-t_{n}$, para $n\geq 1$

$$x'= \lbrace X_{\tau}(\omega), \tau \in \mathbb{T}'\rbrace$$

$x'$ es un proceso estocástico irregularmente espaciado. Una serie de tiempo irregularmente espaciada es una 
realización finita de un proceso estocástico irregularmente espaciado. Note que si  $\Delta_{n+1}=t_{n+1}-t_{n}=1$ 
para $n \geq 1$, el proceso $x'$ es un proceso regularmente espaciado, por tanto esta definición es más general.

\section{Formas de describir un proceso estocástico}
Una forma de describir un proceso estocástico $x$, es especificar la función de distribución conjunta de 
$\lbrace X_{\tau_1},X_{\tau_2}, ..., X_{\tau_n}\rbrace$ para todo $n$, este es llamado 
\emph{punto de vista distribucional}. Por otra parte, se puede describir el proceso proporcionando 
una formula para el valor $X_{\tau}$ para cada punto $\tau$ en términos de una familia de variables aleatorias 
con comportamiento probabilístico conocido, esto hace que podamos ver el proceso como una función de otros 
procesos (o cómo familias de procesos iid); esta forma es llamada \emph{punto de vista construccionista}.

\section{El proceso de medias móviles irregular de primer orden (IMA)}
Teniendo en cuenta el conjunto $\mathbb{T}'= \lbrace t_1,t_2, t_3,... \rbrace$ propuesto anteriormente, 
tal que sus diferencias $\Delta_{n+1}$ para $n\geq 1$ están acotados uniformemente lejos de cero. Ahora, 
sea $m: \mathbb{T}' \mapsto \mathbb{R} $ una función tal que $m(t_n)=0$, para cualquier $t_n \in \mathbb{T}'$. 
A continuación, sea $\Gamma: \mathbb{T}'x\mathbb{T}' \mapsto \mathbb{R}$ una función tal que para cualquier 
pareja $t_n,t_s \in \mathbb{T}'$,

\begin{equation*}
\Gamma (t_n,t_s)=
    \begin{cases}
    \gamma_0& |n-s|=0\\
    \gamma_1,\Delta_{max(n,s)} & |n-s|=1\\
    0 & |n-s|\geq1\\
    \end{cases}
\end{equation*}

Note que $\Gamma$ puede ser representada cómo una matriz diagonal así

\begin{equation}
    \Gamma=
	\begin{bmatrix} 
	\gamma_0 & \gamma_1,\Delta_2 & 0 & 0 & ... \\
	\gamma_1,\Delta_2 & \gamma_0 & \gamma_1,\Delta_3 & 0\\
	0 & \gamma_1,\Delta_3 & \gamma_0 & \gamma_1,\Delta_4 \\
	.& & & & . \\
	.& & & & & .\\
	. & & & & & & .
	\end{bmatrix}
	\quad
	\label{mat1}
\end{equation}

Ser $\Gamma_n$ la truncacion $nxn$ de $\Gamma$ y asumiendo $\gamma_1,\Delta_j\neq 0$, para $j=2,...n$, $\Gamma_n$ 
es definida positiva si $\gamma_0$ y $(\frac{\gamma_1,\Delta_{n+1}}{\gamma_0})^2 \leq 1/4$, para $j=2,...,n$

Esto implica que existe un proceso Gaussiano estacionario $\lbrace X_{t_n},t_n, \tau \in \mathbb{T}\rbrace$, 
único hasta la equivalencia, con media 0 y covarianza $\Gamma$. Este proceso es llamado, proceso de medias móviles
 de primer orden irregularmente espaciado de forma general. A continuación se darán las expresiones particulares 
 de este proceso desde los dos puntos de vista antes definidos.

\subsubsection{El punto de vista distribucional}
En (\ref{mat1}), $\gamma_0$ y $\gamma_{1,\Delta_{n+1}}$, para $n\geq1$, representa la varianza y las covarianzas 
de primer orden respectivamente. Definimos la varianza cómo $\gamma_0=\sigma^2(1+\theta^2)$ y las covarianzas de 
primer orden cómo $\gamma_{1,\Delta_{n+1}}=\sigma^2 \theta^{\Delta_{n+1}}$, $\sigma^2 >0$ y $0<\theta <1$. 
Por tanto, obtenemos el proceso estocástico irregularmente espaciado de primer orden con matriz de covarianzas

\begin{equation}
    \Gamma=
	\begin{bmatrix} 
	1+\theta^2 & \theta^{\Delta_2} & 0 & 0 & ... \\
	\theta^{\Delta_2} & 1+\theta^2 & \theta^{\Delta_3}& 0\\
	0 & \theta^{\Delta_3} & 1+\theta^2 & \theta^{\Delta_4} \\
	.& & & & . \\
	.& & & & & .\\
	. & & & & & & .
	\end{bmatrix}
	\quad
	\label{mat2}
\end{equation}

el cual contiene el modelo de medias moviles convencional cómo caso especial. Este es llamado proceso Gaussiano 
irregular de medias moviles de primer orden.

\subsubsection{El punto de vista construccionista}
Ahora, cómo es usual, especificaremos el proceso IMA cómo función de otros procesos estocásticos. 
Sea $\lbrace \epsilon_{t_n}\rbrace n\geq 1$ variables aleatorias independientes que siguen una distribución 
normal $N(0,\sigma^2 c_n(\theta))$ con $\sigma^2 >0, 0<\theta<1, c_1(\theta)=1+\theta^2$ y

$$
c_n(\theta)=1+\theta^2-\frac{\theta^{2\Delta_n}}{c_{n-1}(\theta)} para, n\geq 2
$$

donde $\Delta_n=t_n-t_{n-1}$. El proceso $\lbrace X_{t_n},t_n \in \mathbb{T'} \rbrace$, es decirse tiene un proceso 
IMA si $X_{t_1}=\epsilon_{t_1}$ y para $n\geq 2 $

\begin{equation}
    X_{t_n}= \epsilon_{t_n}+\frac{\theta^{\Delta_n}}{c_{n-1}(\theta)}\epsilon_{t_{n-1}}
\end{equation}
Decimos que $\lbrace X_{t_n},t_n \in \mathbb{T'} \rbrace$ es un proceso IMA con media 
$\mu$ si $\lbrace X_{t_n}-\mu,t_n \in \mathbb{T'} \rbrace$ es un proceso IMA