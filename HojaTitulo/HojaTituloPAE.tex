% HojaTituloPAE contiene toda la informaci\'{o} acerca de la presentaci\'{o}n del Trabajo de Grado
\thispagestyle{empty}

\begin{center}
\begin{figure}
\centering
\epsfig{file=HojaTitulo/Fig_HojaTitulo/LogoPUC.jpg,scale=0.4}
\end{figure}\vspace*{2.0cm}

%Reemplace este t\'{\i}tulo por el de su Trabajo de Grado
\textbf{\huge \LaTeX\ template: Title}\vspace*{4.0cm}

%Aqui va los nombres y apellidos completos de los autores
\Large\textbf{Wilmar Sepulveda Herrera}\vspace*{4.0cm}

\small Pontificia Universidad Cat\'{o}lica de Chile

Faculty of Mathematics, Department of Statistics

Santiago de Chile, Chile

%Modifique fecha de ser necesario
2019
\end{center}

\newpage{\pagestyle{empty}\cleardoublepage}

\newpage

\thispagestyle{empty}

\begin{center}
%Reemplace este t\'{\i}tulo por el de su Trabajo de Grado
\textbf{\huge \LaTeX\ template: Title}\vspace*{1.0cm}

by\vspace*{0.5cm}

%Aqui va los nombres y apellidos completos de los autores
\Large\textbf{C\'{e}sar Andr\'{e}s Ojeda Echeverri}\vspace*{1.0cm}

\small A dissertation submitted in partial fulfillment\\
of the requirements for the degree of:\vspace*{1.0cm}

%Modifique seg\'{u}n el g\'{e}nero
\textbf{Doctor of Philosophy}\vspace*{0.5cm}

in\vspace*{0.5cm}

\textbf{Statistics}\vspace*{1.5cm}

We accept this thesis as\\
conforming to the required standard\vspace*{1.0cm}

\rule{10cm}{1pt}\vspace*{0.7cm}
\rule{10cm}{1pt}\vspace*{0.7cm}
\rule{10cm}{1pt}\vspace*{2.5cm}

Pontificia Universidad Cat\'{o}lica de Chile

Facultad de Matem\'{a}ticas, Departamento de Estad\'{\i}stica

Santiago de Chile, Chile

%Modifique fecha de ser necesario
2019
\end{center}

\newpage

\thispagestyle{empty}

\vspace*{1cm}

\textbf{\LARGE Dedication}\vspace*{4.0cm}

\begin{flushright}
\begin{minipage}{8cm}
To my wife, with love.
\end{minipage}
\end{flushright}

\newpage{\pagestyle{empty}\cleardoublepage}

\newpage

\thispagestyle{empty}

\vspace*{1cm}

\textbf{\LARGE Acknowledgments}\vspace*{1.0cm}

I am deeply grateful to\ldots

\newpage{\pagestyle{empty}\cleardoublepage}

\newpage

\vspace*{1cm}

\textbf{\LARGE Abstract}\vspace*{1cm}
\addcontentsline{toc}{chapter}{\numberline{}Abstract}

El resumen es una presentaci\'{o}n abreviada y precisa. Se debe usar una extensi\'{o}n m\'{a}xima de 12 renglones. Se recomienda que este resumen sea anal\'{\i}tico, es decir, que sea completo, con informaci\'{o}n cuantitativa y cualitativa, generalmente incluyendo los siguientes aspectos: objetivos, dise\~{n}o, lugar y circunstancias (u objetivo del estudio), principales resultados, y conclusiones. Al final del resumen se deben usar palabras claves tomadas del texto (m\'{\i}nimo 3 y m\'{a}ximo 7 palabras), las cuales permiten la recuperaci\'{o}n de la informaci\'{o}n.\vspace*{0.5cm}

\textbf{\small Palabras clave: palabras clave en espa\~{n}ol (m\'{a}ximo 10 palabras, preferiblemente seleccionadas de las listas internacionales que permitan el indizado cruzado)}\vspace*{1cm}

\textbf{\LARGE Resumen}\vspace*{1cm}

Es el mismo resumen pero traducido al ingl\'{e}s. Se debe usar una extensi\'{o}n m\'{a}xima de 12 renglones. Al final del Abstract se deben traducir las anteriores palabras clave tomadas del texto (m\'{\i}nimo 3 y m\'{a}ximo 7 palabras), llamadas keywords. Es posible incluir el resumen en otro idioma diferente al espa\~{n}ol o al ingl\'{e}s, si se considera como importante dentro del tema tratado en la investigaci\'{o}n, por ejemplo: un trabajo dedicado a problemas ling\"{u}\'{\i}sticos del mandar\'{\i}n seguramente estar\'{\i}a mejor con un resumen en mandar\'{\i}n.\vspace*{0.5cm}

\textbf{\small Keywords: palabras clave en ingl\'{e}s (m\'{a}ximo 10 palabras, preferiblemente seleccionadas de las listas internacionales que permitan el indizado cruzado)}

\newpage{\pagestyle{empty}\cleardoublepage}