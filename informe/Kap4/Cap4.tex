\chapter{Conclusiones}

A continuaci\'{o}n se presentan los requisitos necesarios para la realizaci\'{o}n del Trabajo de Grado en el Programa Acad\'{e}mico de Estad\'{\i}stica basados en la reglamentaci\'{o}n establecida por el Consejo Superior de la Universidad del Valle.

\section{Reglamento estudiantil}

El Acuerdo 009 de 1997 emanado por el Consejo Superior
\begin{itemize}
\item Reglamenta las actividades acad\'{e}micas de los estudiantes regulares pertenecientes a los Progr\'{a}mas Acad\'{e}micos de Pregrado.

\item En el cap\'{\i}tulo XIV reglamenta los trabajos de grado
\end{itemize}

\section{De los trabajos de grado}

El Acuerdo No. 009 del 13 de noviembre de 1997, emanado por el Consejo Superior establece en su art\'{\i}culo 90 que:\\

\textbf{Art\'{\i}culo 90}: En todos los Programas Acad\'{e}micos de la Universidad,
se exigir\'{a} como requisito parcial para la obtenci\'{o}n del t\'{\i}tulo, un Trabajo de Grado, el cual podr\'{a} tener diferentes modalidades: Monograf\'{\i}a,
Proyecto, Pasant\'{\i}a, Pr\'{a}ctica, Ensayo, Traducci\'{o}n Cr\'{\i}tica u otras
aprobadas por el Consejo Acad\'{e}mico. Cada modalidad depender\'{a} de los objetivos del Programa Acad\'{e}mico, del perfil profesional del egresado, del nivel de exigencia que el Programa Acad\'{e}mico defina para esta asignatura y de los intereses del estudiante.\\\\ El Trabajo de Grado puede orientarse a la sistematizaci\'{o}n de conocimientos, a la formulaci\'{o}n y soluci\'{o}n de problemas de investigaci\'{o}n, a la definici\'{o}n y dise\~{n}o de proyectos destinados a la aclaraci\'{o}n de aspectos pr\'{a}cticos de diferente orden, al dise\~{n}o, realizaci\'{o}n y evaluaci\'{o}n de proyectos de intervenci\'{o}n en el \'{a}rea profesional o a la actividad pr\'{a}ctica en la soluci\'{o}n de problemas en las
respectivas disciplina.\\\\\
Le corresponde al Consejo de Facultad definir aquellos aspectos comunes de reglamentaci\'{o}n de sus Programas Acad\'{e}micos.\\


\subsection{Reglamentaciones por facultad, Distinciones}

\begin{itemize}
\item \textbf{ Par\'{a}grafo 1.} Los Comit\'{e}s de Curr\'{\i}culo de las diferentes
Facultades, definir\'{a}n aquellos aspectos de la reglamentaci\'{o}n
comunes a sus Programas y, con base en ellos, cada Programa
Acad\'{e}mico establecer\'{a} su reglamento espec\'{\i}fico.

\item \textbf{Par\'{a}grafo 2.} En la Universidad se calificar\'{a}n como ``meritorios'' \'{o} ``laureados'', aquellos Trabajos de Grado que, de acuerdo con las
reglamentaciones de las respectivas Facultades, alcancen los
niveles de excelencia requeridos para la asignaci\'{o}n de tales
calificaciones.
\end{itemize}

\subsection{Duraci\'{o}n}
\textbf{Art\'{\i}culo 91.} Todo estudiante tendr\'{a} un plazo hasta de dos (2)
semestres para concluir su Trabajo de Grado. Este plazo se
contabilizar\'{a} a partir de la primera matr\'{\i}cula de la \emph{asignatura} ``Trabajo
de Grado''.
\begin{itemize}
\item\textbf{ Par\'{a}grafo.} Por causa debidamente justificada, el Comit\'{e} del
Programa Acad\'{e}mico podr\'{a} prorrogar el plazo se\~{n}alado hasta por
dos (2) semestres.
\item\textbf{ Art\'{\i}culo 92.} En los dos (2) primeros semestres de que habla el
Art\'{\i}culo anterior, el estudiante matricular\'{a} la asignatura ``Trabajo
de Grado'', o su equivalente, y en los subsiguientes, la asignatura
``Continuaci\'{o}n del Trabajo de Grado''.
\end{itemize}

\subsection{Exenci\'{o}n de matr\'{\i}cula durante la continuaci\'{o}n de trabajo de
grado}
\textbf{Par\'{a}grafo} Si el estudiante matricula \'{u}nicamente la asignatura
``Continuaci\'{o}n del Trabajo de Grado'', podr\'{a} cancelar el valor
correspondiente al 25\% del salario m\'{\i}nimo mensual vigente por
concepto de matr\'{\i}cula financiera. El estudiante que deba matricular
otras asignaturas o \'{e}l que encontr\'{a}ndose a paz y salvo con su
Programa Acad\'{e}mico desee tomar otras asignaturas, deber\'{a} cancelar
el valor total de la matr\'{\i}cula. El Director del Programa velar\'{a} por el
cumplimiento de esta norma y la secci\'{o}n de Registro Acad\'{e}mico
verificar\'{a} si esto se efectu\'{o}.

\subsection{Condiciones de grado}
\textbf{Art\'{\i}culo 93}. Para la obtenci\'{o}n del grado, el estudiante debe
matricular en forma sucesiva los semestres que sean necesarios para
la culminaci\'{o}n, presentaci\'{o}n, sustentaci\'{o}n y aprobaci\'{o}n del Trabajo de
Grado.
\begin{itemize}

\item \textbf{Par\'{a}grafo 1.} Si el estudiante ha cumplido con todos los requisitos
para grado, podr\'{a} recibir su t\'{\i}tulo en las fechas previstas para
ello, aunque no haya finalizado el per\'{\i}odo acad\'{e}mico en el cual
se encuentra matriculado.

\item \textbf{Par\'{a}grafo 2.} Si el estudiante no realiza la matr\'{\i}cula de manera
sucesiva, deber\'{a} solicitar reingreso.

\item \textbf{Par\'{a}grafo 3.} Si el estudiante no present\'{o} el Trabajo de Grado en
los plazos establecidos o si el resultado final del Trabajo de Grado
no es aprobatorio, podr\'{a} solicitar su reingreso al Programa
Acad\'{e}mico. En estos casos, el Comit\'{e} de Programa Acad\'{e}mico
definir\'{a} cu\'{a}les asignaturas adicionales al Trabajo de Grado
deber\'{a} matricular.

\item \textbf{Par\'{a}grafo 4.} Si por segunda vez se vence el plazo m\'{a}ximo
previsto de cuatro (4) semestres y no se ha aprobado el Trabajo
de Grado, el estudiante pierde el derecho a optar por el t\'{\i}tulo y el
Director de Programa Acad\'{e}mico notificar\'{a} a las instancias
respectivas la calificaci\'{o}n de ``no aprob\'{o}''.
\end{itemize}


\section{Reglamentaci\'{o}n en la Facultad de Ingenier\'{\i}a: Resoluci\'{o}n No 074 de junio 21 de 2011}

\subsection{Resoluci\'{o}n No 074 de junio 21 de 2011, Consejo de Facultad}
Reglamenta el Trabajo de Grado para la Facultad de Ingenier\'{\i}a.
\begin{itemize}
\item \textbf{Art\'{\i}culo 2.} El Trabajo de Grado podr\'{a} tener diferentes
modalidades tales como Monograf\'{\i}a, Trabajo de investigaci\'{o}n e
innovaci\'{o}n, Pasant\'{\i}as, Creaci\'{o}n de empresa o Cursos en
postgrado.

\item \textbf{Par\'{a}grafo 1.} Los Comit\'{e}s de Programa Acad\'{e}mico reglamentar\'{a}n y definir\'{a}n, con base en la estructura curricular y prop\'{o}sitos de
formaci\'{o}n de los programas acad\'{e}micos, las modalidades de Trabajo de Grado a las que pueden optar sus estudiantes.
\end{itemize}

\subsection{Trabajo de Investigaci\'{o}n e innovaci\'{o}n}
\begin{itemize}
\item Hasta la fecha la modalidad aceptada en Estad\'{\i}stica es la
denominada ``Trabajo de investigaci\'{o}n e Innovaci\'{o}n'', que se
define as\'{\i}:
\item Presentaci\'{o}n formal del resultado de un proceso de exploraci\'{o}n,
descripci\'{o}n, correlaci\'{o}n, explicaci\'{o}n o compresi\'{o}n de sujetos,
objetos o fen\'{o}menos. Debe seguir la metodolog\'{\i}a cient\'{\i}fica y
cubrir las diversas etapas del proceso de investigaci\'{o}n, para crear
nuevo conocimiento o plantear soluciones factibles. Esta actividad
debe adelantarse con el aval de un grupo de investigaci\'{o}n o en el
marco de un proyecto en particular, buscando una efectiva
orientaci\'{o}n del trabajo.
\end{itemize}


\subsection{El informe final}
\textbf{Par\'{a}grafo 2}. Todas las modalidades de Trabajo de Grado concluyen
en un informe escrito y una sustentaci\'{o}n p\'{u}blica conforme a lo
establecido por esta Resoluci\'{o}n exceptuando la excelencia
acad\'{e}mica.

\subsection{Asignaturas que conforman el trabajo de grado en programas de
ingenier\'{\i}a}

\textbf{Art\'{\i}culo 6.} El Trabajo de Grado para un Programa Acad\'{e}mico en
Ingenier\'{\i}a tendr\'{a} m\'{\i}nimo siete (7) cr\'{e}ditos distribuidos en tres (3)
asignaturas, as\'{\i}:
\begin{itemize}
\item Anteproyecto o su equivalente Un (1) Cr\'{e}dito m\'{\i}nimo
\item Trabajo de Grado I o su equivalente Dos (2) Cr\'{e}ditos m\'{\i}nimo
\item Trabajo de Grado II o su equivalente Cuatro (4) Cr\'{e}ditos m\'{\i}nimo
\end{itemize}

\subsection{Asignaturas que conforman el trabajo de grado en el programa
acad\'{e}mico de Estad\'{\i}stica}

La Resoluci\'{o}n No. 079 del 6 de Junio del 2002 establece:\\

\textbf{Art\'{\i}culo 6.} El programa acad\'{e}mico de Estad\'{\i}stica, c\'{o}digo 3752, tiene
una duraci\'{o}n de 10 semestres, modalidad presencial y jornada diurna
y exige como m\'{\i}nimo 166 cr\'{e}ditos.
\begin{itemize}
\item\textbf{ Par\'{a}grafo 7.} El Trabajo de Grado est\'{a} conformado por tres
asignaturas, Estad\'{\i}stica Aplicada IV, Trabajo de Grado I y Trabajo
de Grado II. De estos, el curso de Estad\'{\i}stica Aplicada IV est\'{a}
reservado para que en el desarrollo del mismo se realice el
anteproyecto del Trabajo de Grado.
\end{itemize}

\section{Distinciones en Ingenier\'{\i}a: Resoluci\'{o}n No 085 de Mayo 26 de 2009}

\subsection{Resoluci\'{o}n No 085 de 2009, Consejo de Facultad}
Establece los criterios y el procedimiento para la calificaci\'{o}n de
Meritorio o Laureado de un Trabajo de Grado de Pregrado en la
Facultad de Ingenier\'{\i}a.

\begin{itemize} 
\item \textbf{Art\'{\i}culo 2.} Para optar a la calificaci\'{o}n de Meritorio o Laureado,
son requisitos indispensables que el Trabajo de Grado se haya
ejecutado en el tiempo aprobado en el cronograma del
anteproyecto, sin superar lo establecido en la estructura curricular
del programa para la terminaci\'{o}n del Trabajo de Grado, con las
excepciones de fuerza mayor debidamente certificadas y
autorizadas por el Comit\'{e} de Programa, y que el Trabajo de Grado
haya obtenido como m\'{\i}nimo la siguiente calificaci\'{o}n num\'{e}rica:
\begin{itemize}
\item Trabajo de Grado Meritorio 4.5 (Cuatro punto cinco)
\item Trabajo de Grado Laureado 5.0 (Cinco punto cero)
\end{itemize}
\end{itemize}

\section{Recomendaciones}
\begin{itemize}
\item El trabajo de grado no debe exceder un tama\~{n}o m\'{a}ximo de 80
p\'{a}ginas (Resoluci\'{o}n 074 de 2011), incluyendo anexos.
\item El anteproyecto de grado, en consecuencia, no debe exceder un
tama\~{n}o m\'{a}ximo de 20 p\'{a}ginas, incluyendo anexos.
\item El periodo de ejecuci\'{o}n del proyecto es de m\'{a}ximo un a\~{n}o.
\end{itemize}