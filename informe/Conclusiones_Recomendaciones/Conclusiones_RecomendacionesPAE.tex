\chapter{Conclusiones}

A lo largo de este trabajo, se ha enfatizado la importancia tanto práctica como teórica de abordar los procesos estocásticos con espaciamiento irregular. Se ha desarrollado un modelo autorregresivo irregular (IAR) que permite estructuras de autocorrelación tanto negativas como positivas, lo cual amplía las aplicaciones potenciales de estos modelos. El modelo IAR propuesto ha demostrado satisfacer estas características de manera efectiva. \\

Además, se han encontrado estimadores de máxima verosimilitud y bootstrap para el modelo propuesto, los cuales se han demostrado insesgados, consistentes y de fácil implementación. Estas estimaciones han sido comparadas con un modelo existente llamado modelo CIAR, y se ha observado que los estimadores propuestos son igual de buenos o incluso mejores en algunos casos particulares. \\

También se destaca la relevancia práctica de este modelo al analizar un escenario de aplicación en el cual se obtiene un parámetro negativo, que fue el enfoque principal de este proyecto. Esto resalta la importancia práctica del modelo propuesto y abre la puerta para que otros investigadores lo utilicen en sus problemas específicos.\\

Un trabajo futuro para este tipo de modelos es proponer estimaciones bayesianas, las cuales podrían mejorar la precisión de los estimadores. Además, se sugiere realizar una comparación más exhaustiva con otros modelos, como el CIAR, para encontrar posibles equivalencias o demostrar definitivamente que son modelos diferentes.\\

Los investigadores pueden seguir esta metodología para explorar funciones más complejas de $\phi_n$ y generar soluciones adicionales para este interesante y apasionante problema.