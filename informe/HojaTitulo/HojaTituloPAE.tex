% HojaTituloPAE contiene toda la informaci\'{o} acerca de la presentaci\'{o}n del Trabajo de Grado
\thispagestyle{empty}

\begin{center}
\begin{figure}
\centering
\epsfig{file=HojaTitulo/Fig_HojaTitulo/Univalle.jpg,scale=0.15}
\end{figure}\vspace*{1.5cm}

%Reemplace este t\'{\i}tulo por el de su Trabajo de Grado
\textbf{\huge Procesos estocásticos irregularmente espaciados: Una variación del modelo autorregresivo irregular de primer orden IAR}\vspace*{3.5cm}

%Aqui va los nombres y apellidos completos de los autores
\Large\textbf{Wilmar Sepulveda Herrera}\vspace*{3.5cm}

\small Universidad del valle

Facultad de Ingeniería, Escuela de Estadística

Cali, Valle del Cauca

%Modifique fecha de ser necesario
2023
\end{center}

\newpage{\pagestyle{empty}\cleardoublepage}

\newpage

\thispagestyle{empty}

\begin{center}
%Reemplace este t\'{\i}tulo por el de su Trabajo de Grado
\textbf{\huge Procesos estocásticos irregularmente espaciados}\vspace*{1.0cm}

by\vspace*{0.5cm}

%Aqui va los nombres y apellidos completos de los autores
\Large\textbf{Wilmar Sepulveda Herrera}\vspace*{1.0cm}

\small Trabajo de investigación para optar por el título de:\vspace*{1.0cm}

%Modifique seg\'{u}n el g\'{e}nero
\textbf{Magister en}\vspace*{0.5cm}

En\vspace*{0.5cm}

\textbf{Estadística}\vspace*{1.5cm}

Aceptamos esta tesis\\
Conforme a los requerimientos de la norma\vspace*{1.0cm}

\rule{10cm}{1pt}\vspace*{0.7cm}
\rule{10cm}{1pt}\vspace*{0.7cm}
\rule{10cm}{1pt}\vspace*{2.3cm}

\small Universidad del valle

Facultad de Ingeniería, Escuela de Estadística

Cali, Valle del Cauca

%Modifique fecha de ser necesario
2023
\end{center}


\newpage

\thispagestyle{empty}

\vspace*{1cm}

\textbf{\LARGE Agradecimientos}\vspace*{1.0cm}

Estoy agradecido con el director de esta tesis, el profesor Cesar Ojeda que siempre estuvo dispuesto a ayudarme, y con mi novia, que siempre estuvo apoyandome incluso cuando pensé en abandonar. Muchas gracias.

\newpage{\pagestyle{empty}\cleardoublepage}

\newpage

\vspace*{1cm}

\textbf{\LARGE Resumen}\vspace*{1cm}
\addcontentsline{toc}{chapter}{\numberline{}Abstract}

Este trabajo se centra en el estudio de los procesos estocásticos irregularmente espaciados, los cuales son de gran importancia en diversos contextos pero a menudo pasan desapercibidos. Aunque se han propuesto enfoques previos para abordar este fenómeno, como el modelo Autorregresivo Irregular de primer orden (IAR), estos modelos solo permiten autocorrelaciones positivas. En esta investigación, se busca generalizar el modelo IAR extendiéndolo para incluir autocorrelaciones negativas.

Se propone una función específica que resuelve este problema y se deja abierta la posibilidad de proponer funciones más robustas en futuros estudios. Además, se desarrollaron estimadores basados en la máxima verosimilitud y bootstrap para el modelo propuesto, y se evaluaron sus propiedades mediante simulaciones.

Finalmente, se aplicó el modelo a un conjunto de datos reales para analizar su relevancia práctica. Los resultados demuestran que el modelo propuesto puede ser utilizado por investigadores para abordar una amplia gama de problemas relacionados con procesos estocásticos.
\textbf{\small Palabras clave: Procesos estocásticos, Series de tiempo, Tiempos irregulares, Máxima verosimilitud, Bootstrap, Simulación.}\vspace*{1cm}

\textbf{\LARGE Abstract}\vspace*{1cm}


This work focuses on the study of irregularly spaced stochastic processes, which are of great importance in various contexts but often go unnoticed. Although previous approaches have been proposed to address this phenomenon, such as the first-order Irregular Autoregressive (IAR) model, these models only allow for positive autocorrelations. In this research, we aim to generalize the IAR model by extending it to include negative autocorrelations.

A specific function is proposed to solve this problem, and the possibility of proposing more robust functions in future studies is left open. Furthermore, maximum likelihood and bootstrap estimators were developed for the proposed model, and their properties were evaluated through simulations.

Finally, the model was applied to a real dataset to assess its practical relevance. The results demonstrate that the proposed model can be used by researchers to address a wide range of problems related to stochastic processes.

\textbf{\small Keywords: Stochastic processes, Time series, Irregular times, Maximum likelihood, Bootstrap, Simulation.}

\newpage{\pagestyle{empty}\cleardoublepage}