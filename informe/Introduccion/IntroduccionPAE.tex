\pagenumbering{arabic}
\chapter{Introducci\'{o}n}
Los procesos estocásticos, particularmente el análisis de series de 
tiempo, hacen parte de las metodologías estadísticas más utilizadas
en la actualidad, ya sea para automatizar procesos de Machine Learning
enfocados al pronóstico de valores futuros o para conocer las
características propias de un fenómeno y su estructura de 
autocorrelación. Dicha función de autocorrelación puede ser 
positiva, con importancia en muchos contextos cómo la demografía
\citep[p.ej.]{novoseltseva2019spatial} o la explicación de 
fenómenos ecológicos \citep[p.ej.]{yang2019predictability}, 
entre otras. Las autocorrelaciones negativas también tienen 
mucha importancia práctica, por ejemplo; en el sector financiero 
para conocer el comportamiento de los rendimientos de acciones 
\citep[p.ej.]{kuttu2017feedback} y en la biología 
\citep[p.ej.]{rindorf2020periodic}. Bajo este contexto, los 
procesos estocásticos son importantes en diversos campos del 
conocimiento para darnos a conocer las características de la
dependencia temporal de un evento de interés y con ella tomar 
decisiones importantes.


La teoría de series temporales ha proporcionado diversos métodos para 
modelar la autocorrelación de una secuencia de interés, entre ellos, 
los modelos autorregrevos integrados de medias móviles (ARIMA), 
sin embargo, estos métodos asumen que las observaciones son tomadas
regularmente en el tiempo, supuesto que en muchos casos no es válido,
pues la presencia de valores faltantes o la misma naturaleza de las 
observaciones, pueden inducir a que la serie de tiempo sea irregularmente
espaciada \citep{elorrieta2019discrete}. Es común encontrar series de 
tiempo irregularmente espaciadas en diversas aplicaciones, para 
diferentes áreas del conocimiento: sistemas caóticos, medicina, 
imágenes satelitales y calidad del aire; ver por ejemplo 
\cite{shamsan2020intrinsic}, \cite{liu2019comparison},
\cite{ghaderpour2020change} y 
\cite{dilmaghani2007harmonic}, entre otros.



En la actualidad, la literatura proporciona varios métodos para
modelar series de tiempo irregularmente espaciadas, algunos 
autores como \cite{adorf1995interpolation} transforman 
las series de tiempo irregulares en regulares vía interpolación,
método que puede generar sesgos;  \cite{robinson1977estimation} 
aborda el problema tratando la serie de tiempo como una realización
discreta de un proceso estocástico de tiempo continuo, aproximación
que también presenta problemas. Recientemente, 
\cite{eyheramendy2018irregular} proponen el Modelo Autorregresivo 
Irregular  (\emph{IAR}) de primer orden, el cual tiene la limitación no 
de aceptar valores de negativos en su función de autocorrelación (ACF); 
este problema fue resuelto por los autores en un estudio posterior,
en el cual proponen el modelo Autorregresivo Complejo Irregular 
(\emph{CIAR}). \citep{elorrieta2019discrete}
 
 
 
En la misma línea \ {ojeda:inpress-a} proponen el modelo de medias móviles irregular (\emph{IMA}), el cual bajo normalidad,
es estrictamente estacionario, mientras que, teniendo en cuenta otros 
supuestos distribucionales, se comporta cómo un proceso débilmente 
estacionario; en ambos casos, los autores evalúan las propiedades
estadísticas de los modelos y proponen métodos de estimación, 
encontrando muy buenos resultados. Sin embargo, el modelo IMA 
tiene una desventaja, en términos de que su ACF es únicamente 
positiva, siendo incapaz de detectar autocorrelaciones negativas 
y dejándolo muy limitado debido a la importancia practica que estas 
suponen; problema que no tienen los modelos de medias móviles clásicos.
El propósito de este trabajo es modificar el modelo IMA de modo que 
permita modelar estructuras de autocorrelación positivas y negativas
sin necesidad de irse al plano de los números complejos.
 
\section{Objetivos}
\subsection{Objetivo general}
Proponer un modelo autorregresivo irregular de primer orden que 
permita modelar estructuras de autocorrelación positivas y negativas
subyacentes de un proceso estocástico.
\subsection{Objetivos específicos}
\begin{itemize}
    \item Encontrar estimadores para el modelo propuesto.
    \item Evaluar las propiedades del modelo propuesto y sus estimaciones vía simulación.
    \item Ajustar el modelo propuesto a un conjunto de datos para valorar su comportamiento en ejercicios de aplicación.
\end{itemize}