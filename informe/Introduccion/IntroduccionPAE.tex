\pagenumbering{arabic}
\chapter{Introducci\'{o}n}


Los procesos estocásticos, y en particular el análisis de series de tiempo,
son metodologías estadísticas ampliamente utilizadas en la actualidad. 
Estas técnicas se aplican comúnmente en el pronóstico de valores futuros
y en el estudio de las características de un fenómeno, incluyendo su 
estructura de autocorrelación. La función de autocorrelación puede ser 
positiva, lo que es importante en muchos contextos, como la demografía 
(por ejemplo, \cite{novoseltseva2019spatial}) o la explicación de fenómenos 
ecológicos (por ejemplo, \cite{yang2019predictability}). También pueden darse 
autocorrelaciones negativas, que son relevantes en el sector 
financiero para entender el comportamiento de los rendimientos de 
acciones (por ejemplo, \cite{kuttu2017feedback}) y en la biología 
(por ejemplo, \cite{rindorf2020periodic}). En este contexto, los procesos 
estocásticos son importantes en diversos campos del conocimiento para 
caracterizar la dependencia temporal de un evento de interés y tomar 
mejores decisiones.


La teoría de series temporales es ampliamente utilizada en la actualidad
para modelar la autocorrelación de una secuencia de interés. Uno de 
los métodos más populares es el modelo autorregresivo integrado de 
medias móviles (ARIMA). Sin embargo, este y otros métodos similares 
asumen que las observaciones son tomadas regularmente en el tiempo, 
lo cual no siempre es el caso en la práctica. Esto puede deberse a la 
presencia de valores faltantes o a la naturaleza de las observaciones, 
lo que puede hacer que la serie de tiempo sea irregularmente espaciada 
\citep{elorrieta2019discrete}. De hecho, es común encontrar series de 
tiempo con este tipo de irregularidades en diversas aplicaciones, como
en sistemas caóticos, medicina, imágenes satelitales y calidad del 
aire, entre otros \citep{shamsan2020intrinsic, liu2019comparison, 
ghaderpour2020change, dilmaghani2007harmonic}.



En la literatura actual, existen diversos métodos para modelar series de 
tiempo que no son tomadas regularmente en el tiempo. Algunos autores, 
como \cite{adorf1995interpolation}, abordan este problema transformando 
las series irregulares en regulares a través de la interpolación, pero 
este enfoque puede generar sesgos. Otros, como \cite{robinson1977estimation},
tratan la serie de tiempo como una realización discreta de un proceso 
estocástico de tiempo continuo, lo cual también presenta problemas. 
Recientemente,  \cite{ojedaIMA} propusieron el modelo de medias móviles irregulares 
(\emph{IMA}). Bajo normalidad, este modelo es estrictamente estacionario, 
mientras que, considerando otros supuestos distribucionales, se comporta como un proceso débilmente estacionario. 
 
 

En esta línea, Eyheramendy et al. \cite{eyheramendy2018irregular} proponen el Modelo 
Autorregresivo Irregular (IAR) de primer orden. Sin embargo, este modelo tiene una limitación 
importante: no acepta valores negativos en su función de autocorrelación (ACF). Para solucionar 
este problema, los autores presentaron el Modelo Autorregresivo Complejo Irregular (CIAR) en un 
estudio posterior \citep{elorrieta2019discrete}. En ambos casos, se han evaluado las propiedades 
estadísticas de los modelos y se han propuesto métodos de estimación que han obtenido buenos 
resultados. No obstante, tanto el modelo IAR como el IMA tienen una desventaja: su función de 
autocorrelación solo puede ser positiva, lo que limita su capacidad para detectar 
autocorrelaciones negativas, que suelen ser importantes en la práctica. Este problema no está 
presente en los modelos ARIMA clásicos. Por lo tanto, el objetivo de este trabajo es 
modificar el modelo IAR para que pueda modelar tanto autocorrelaciones positivas como negativas 
sin necesidad de recurrir al plano complejo.
 
\section{Objetivos}
\subsection{Objetivo general}
Desarrollar un modelo autorregresivo irregular de primer orden capaz de 
modelar estructuras de autocorrelación subyacentes con valores tanto positivos como negativos.
\subsection{Objetivos específicos}
\begin{itemize}
    \item Encontrar estimadores por máxima verosimilitud y bootstrap para el modelo propuesto.
    \item Evaluar las propiedades del modelo propuesto y sus estimaciones vía simulación.
    \item Aplicar el modelo propuesto a un conjunto de datos reales y evaluar su capacidad para capturar la dinámica 
          de la serie de tiempo y realizar predicciones precisas.
\end{itemize}